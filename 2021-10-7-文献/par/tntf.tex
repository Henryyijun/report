

\par Li等人提出了一种两级非平稳紧框架系统(TNTF)\cite{li2021regularization},作为图像恢复模型的正则项,该紧框架系统可以有效捕获图像的一阶和二阶信息。文中提出,在对图像分解时,第一级使用DHF\cite{li2016adaptive}的高通滤波器进行分解,第二级使用DCT紧框架对第一级的低通结果进行分解,具体的图像恢复模型为

\begin{equation}
	\underset{u}{min} \{ \mathcal{F}(u) + \mathcal{G}_{1}(u) + \mathcal{G}_{2}(u)\}
\end{equation}
其中 $u$是待求解图像,$\mathcal{F}(u)$为保真项,$\mathcal{G}_{1}(u)$ 为使用DHF高通滤波器的正则化项,$\mathcal{G}_{2}(u)$是DCT紧框架对第一级低通结果进行分解的正则化项,分别使用$l_2$范数和$l_1$范数的形式进行正则化。该文中实验表明,其设计的两级非平稳紧框架系统可以有效的提取图像的一阶和二阶信息,图像恢复质量比其他正则项方法有所提升,证明了所提出的图像恢复模型的有效性。

