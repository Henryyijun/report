%!TEX program = xelatex
% 完整编译: xelatex -> bibtex -> xelatex -> xelatex
\documentclass[lang=cn,11pt,a4paper,cite=numbers]{elegantpaper}

\title{pMRI文献综述}
\author{Henry}
%\institute{\href{https://elegantlatex.org/}{Elegant\LaTeX{} 项目组}}
\date{\zhtoday}
% 本文档命令
\usepackage{array}

\newcommand{\ccr}[1]{\makecell{{\color{#1}\rule{1cm}{1cm}}}}
\begin{document}

\maketitle
\section{背景}
\par 磁共振成像(Magnetic Resonance Imaging,MRI)是一项无创的医学成像技术,该成像技术不需要使用电离辐射,因此被扫描的病人不会受到辐射的侵害,并且它可以得到分辨率和质量较高的病理图像,这些特性使得人体内一些器官结构、血管结构和其他生理特征都可以得到较好的可视化,因此MRI在临床上得到了广泛的应用。但是MRI不同于计算机断层扫描(Computed Tomography,CT)等方法具有成像时间快的特性,MRI该技术最大的障碍就是其成像时间过于缓慢,并且MRI设备一般都是比较密闭的空间,使得病人在接受扫描时可能因为空间过于密闭而导致的无法呼吸、幽闭恐惧症等。因此缩短成像时间,提高成像质量是MRI技术的主要问题所在,并行磁共振成像 \cite{deshmane2012parallel}(parallel Magnetic Resonance Imaging,pMRI)技术是提高MRI速度的主要方法之一,其使用一组线圈,通过获取具有不同空间灵敏度的欠采样的数据去加速成像速度,现已经在临床得到了广泛应用,但是该方法也受限于灵敏度信息。

\section{国内外研究现状}
\par 核磁共振成像技术主要分为两个步骤,分别是数据信号采集和图像重建,图像重建主要依靠欠采样的数据,从而重建出较为完整的图像,但是受限于MRI设备和采样定理的缘故,采集数据耗费大量时间,而且也容易因为病人运动等产生图像伪影。因此众多学者致力于MRI快速成像和重建的研究中,MRI图像重建方法主要分为三类:
\par 第一类,基于图像域的重建方法,该方法需要明确知道MRI设备中线圈灵敏度等信息,通过灵敏度等信息将该重建问题建模为一个逆问题,可以通过最小二乘法等算法进行解决。
\par 第二类,基于频率域(k-空间)的重建方法,由于灵敏度等信息隐式的存在于k-空间数据中,因此在重建时不需要显式知道灵敏度,而是将k-空间缺失的数据通过插值得到。
\par 第三类,基于深度学习的重建方法,该方法通过大量的数据,使用网络进行训练,主要使用的网络为卷积神经网络。

\subsection{基于图像域的重建方法}
\par pMRI系统中使用一组线圈阵列同时进行采集数据,线圈阵列由多个线圈组成,用于采集不同空间位置上的信息,越接近线圈的区域上信号会更强,这种不同空间位置导致的信号强弱被称为线圈的灵感度。通常情况下,线圈的灵感度未知,而基于图像域的重建方法首先需要明确知道灵感度,如SENSitivity Encoding(SENSE)\cite{pruessmann1999sense},SENSE在图像域中对伪影进行去除,但SENSE的缺点主要依赖于精确的灵敏度,而灵敏度在现实中却难以获得。因此估计出精确的灵敏度也是SENSE类方法的重要步骤,例如JSENSE\cite{ying2007joint},TSENSE\cite{kellman2001adaptive}和mSENSE\cite{kreitner2006fast},JSENSE算法将MRI图像重建问题转为线圈敏感度和图像的联合估计问题,使用优化迭代算法进行交替求解;TSENSE提出了一种自适应灵敏度估计的方法,用以提高估计的精确度。
\par 由于pMRI线圈的复杂几何结构,导致很难得到精确的线圈灵敏度数据,而且重建得到的图像也很容易受到噪声放大等影响。因此在SENSE类重建方法中,经常使用正则化技术用以改善重建质量,常用的正则化方法有Tikhonov正则化,全变差正则化(Total Variation, TV)和小波正则化等。在文\cite{ye2010computational}中,提出了一种快速求解TV正则化的算法,用以求解pMRI重建模型。虽然TV正则化的方法有助于去除伪影, 但是也有可能产生阶梯状伪影。Total generalized variation(TGV)\cite{bredies2010total}的提出是为了消除TV的一些缺点,从而提高图像质量。 Knoll等人\cite{knoll2012parallel}通过使用二阶TGV进行正则化的方法,用以消除TV正则化可能产生的阶梯状伪影,并且提出了一种基于非线性的重建方法。随着压缩感知(Compressed Sensing, CS)等理论的研究深入,Lustig等人发现一些MRI图像中的像素表示是较为稀疏的,如果使用适当的重建算法,具有稀疏性的MRI图像将可以从任意采样的k-空间数据中得到恢复,因此提出了Sparse-MRI\cite{lustig2007sparse}。该方法也奠定了CS与MRI结合的算法框架,在Sparse-MRI这一框架下,Liang等人提出了一种新的快速成像算法,其将SENSE和Sparce-MRI进行相结合,称为CS-SENSE\cite{liang2009accelerating}。由于使用CS方法可以得到较为稀疏的重建图像,而且其对图像去噪和抑制伪影的效果也较好,因此CS和pMRI相结合的方法被广泛应用。CS-MRI模型对目标图像进行稀疏变换,对变换得到稀疏系数使用$l_0$范数进行约束,但是求解$l_0$范数是NP难问题,因此将$l_0$范数转化为其最优凸近似$l_1$范数,最后使用非线性算法进行求解。
\par CS-MRI或者CS-SENSE模型中,常使用离散小波(Wavelet),紧框架(Framelet)等进行稀疏变换。因为紧框架系统存在的冗余性,可以较好的对目标图像进行稀疏表示,同时其也具有良好的计算性质,例如正交Haar紧框架具有完美重构的性质,因此也被众多研究者所使用。如Li等人,提出一种基于二维正交Haar小波的紧框架系统,称为DHF\cite{li2016adaptive},可用于检测图像水平,垂直和$±45°$方向上的边缘特征,并提出了相应的求解算法。Liu等人也基于紧框架系统,开发了投影迭代软阈值算法(pISTA)\cite{7448403}和其加速版本pFISTA。但是,在文\cite{7448403}中作者并没有证明其收敛准则,而且其现有的单线圈收敛准则不适用于并行成像版本,因此Zhang等人\cite{ZHANG2021101987}基于Liu的工作基础,对pFISTA应用于并行成像进行了收敛性分析,并且证明了应用于SENSE时的最优参数。CS-SENSE在求解pMRI问题时体现了其优越性,但是保证完全恢复的可能性是基本上不存在的,因此在文\cite{chun2015efficient}中提出了一种联合稀疏的pMRI重建模型JS-CS-SENSE,进一步提升CS-SENSE模型的稀疏性,JS-CS-SENSE重建模型同时使用小波正则化和全变差正则化,能够更加有效的提取先验信息,同时也更好的克服了采样系统和稀疏化之间的相互一致性障碍。虽然联合稀疏的重建模型能够较好解决CS-SENSE模型的缺点,但是也提升了计算的复杂性。


\subsection{基于频率域($k$-空间)的重建方法}
\par 基于k-空间的重建方法与图像域方法的主要区别为,不需要明确知道线圈灵感度的信息,其信息隐含在k-空间的数据中。Generalized autocalibrating partially parallel acquisitions(GRAPPA)\cite{griswold2002generalized},是pMRI重建算法中较为著名的其中之一,该算法将pMRI重建问题视作一个缺失值填充问题,将k-空间中缺失点周围的已采样的点进行线性组合,从而估计出缺失值。GRAPPA较为重要的一步就是从k-空间中心一块全采样的数据(Auto Calibraion Signal,ACS)中估计得到插值核,而且还可以从ACS中估计得到粗略的灵敏度信息,因此线圈灵感度的信息隐含于k-空间中。但是GRAPPA重建质量不佳,因此Lustig等人基于GRAPPA的思想,提出了SPIRiT\cite{lustig2010spirit}算法,它使用了与GRAPPA中相同的插值核,然后将问题构建为一个基于k-空间对数据填空的逆问题,并且为了进一步提升重建图像的质量,纳入了图像的先验知识,使用小波正则化方法对问题进行约束,最后使用非线性算法求解。因此,一些学者也发现了SENSE和GRAPPA两者之间的联系, Uecker\cite{uecker2014espirit}等人将GRAPPA和SENSE相结合,将其解约束在子空间中,对Calibration矩阵进行SVD分解,将分解结果的右奇异矩阵的特征向量作为灵敏度信息,最后提出‘soft’SENSE模型,同时也基于CS-MRI的思想,对目标图像使用小波分解提取先验信息,将分解系数用$l_1$范数进行约束。但是ESPIRiT方法仍未考虑灵敏度的相位信息,在\cite{uecker2017estimating}一文中,将ESPIRiT算法应用于物理和虚拟线圈数据中,提出了VCC-ESPIRiT,该算法利用了k-空间数据的共轭对称性,通过计算得到了含有图像绝对相位的线圈灵敏度信息。
\par GRAPPA重建算法的思想类似于插值,在k-空间中存在另一种思想的算法,即基于低秩矩阵填充的方法,该算法将pMRI问题建模为一个低秩矩阵补全问题。文\cite{shin2014calibrationless}中提出了一种结构化低秩矩阵补全的算法SAKE,该算法不需要k-空间中的ACS即可恢复出完整数据。其首先使用k-空间数据进行构建得到block Hankel矩阵,然后通过数据一致性和block Hankel矩阵的低秩性重建出完整图像数据。ALOHA\cite{7547372}算法将pMRI和CS-MRI转换为使用结构化矩阵的k-空间低秩加权矩阵补全问题,基于ALOHA和SPIRiT的算法,Zhang等人提出STDLR-SPIRiT,该算法使用SPIRiT的插值核对k-空间数据进行校正,同时使用小波变换在k-空间水平方向上和垂直方向上进行分解,然后对分解系数使用核范数约束其低秩的性质,但是该算法在保证重建图像质量的同时也牺牲了计算时间。Jingyuan Lyu\cite{8428648}等人证明了自动校准数据和未获取的k-空间数据之间的非线性关系,基于核估计的方法,提出了更为通用的非线性重建框架,相比于传统的GRAPPA,其使用非线性方法对缺失数据进行估计的重建算法更具有鲁棒性。Blaimer\cite{blaimer2014regularization}等人通过相位约束的方法,提出了phase-constraint GRAPPA,其使用虚拟共轭线圈实现,虚拟共轭线圈通过物理线圈的复共轭信号得到,在GRAPPA进行校准的过程,对实际线圈和虚拟线圈数据施加不同的正则化系数,从而得到了比传统GRAPPA更好的重建效果。

\subsection{基于深度学习的重建方法}
\par 近年来,由于机器学习和深度学习等方法的快速发展,深度学习的方法也被应用于磁共振成像重建问题中。深度学习使用大量的数据进行训练,通过挖掘k-空间数据和目标图像之间的关系完成重建。在深度学习等重建方法中,也可以区分为两类算法,分别是基于图像域的深度学习算法和基于k-空间的重建算法。基于图像域的算法中,MoDL\cite{aggarwal2018modl}基于pMRI重建模型是一个逆问题的性质,提出基于卷积神经网络(CNN)正则化的重建模型,该模型使用CNN提出图像的先验信息,使用数值算法和神经网络训练相结合的方法提高该模型的重建质量。Luo等人根据最新提出的PixelCNN+\cite{salimans2017pixelcnn++},提出基于深度贝叶斯估计的重建模型\cite{luo2020mri},该模型通过贝叶斯定理,将MRI问题进行建模,通过最大化概率的方法完成对欠采样k-空间的重建。基于ESPIRiT方法,文\cite{sandino2021accelerating}中提出DL-ESPIRiT, 设计一种基于(2+1)D的时空卷积神经网络,先使用2D卷积核,然后再使用1D卷积核,产生了比直接使用3D卷积神经网络更好的重建效果。
\par 但是从目前来说,基于深度学习的方法仍未得到广泛应用,最主要的原因就是病人的MRI数据难以获取,而且通过训练得到的模型泛化能力不够强,当数据之间差异较大时,模型的重建结果质量较差,同时也会容易到MRI设备参数的影响,

\newpage
\bibliographystyle{unsrt}
\bibliography{reference}

\end{document}
