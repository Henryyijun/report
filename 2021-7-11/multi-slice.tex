%!TEX program = xelatex
% 完整编译: xelatex -> bibtex -> xelatex -> xelatex
\documentclass[lang=cn,11pt,a4paper,cite=numbers]{elegantpaper}

\title{Multi Slice pMRI Model}
\author{Henry}
\usepackage{array}
\newcommand{\ccr}[1]{\makecell{{\color{#1}\rule{1cm}{1cm}}}}
\begin{document}

\maketitle
\section{Multi slice pMRI reconstruction model by 3D regularization} 
\par pMRI设备采集到的是包含多个切片的k-space数据,每个切片数据通过傅里叶逆变换之后都可以得到对应切片的图像。因此我们可以使用3D正则化的方法对多个相近的切片进行约束,以达到更好的重建质量。
\par 第$i$个切片图像的第$l$个线圈数据定义为
\begin{equation}\label{coil}
	g_{il} = PFS_{il}u_i + \eta_{il},l=1,\dots,p
\end{equation}
其中$\eta_{il}$白噪声,$S_{il}$是对应的敏感度矩阵,$F$表示傅里叶变换,$P$为采样矩阵,其对角元素为0或者1,$u_i$是第$i$个切片图像。
\par 将单个切片的所有线圈数据进行组合,可以得到
\begin{equation}
	g_i = QM_i u_i + \eta_i
\end{equation}
其中$g_i$ 是 $g_{il}, l=1,\dots,p$通过堆叠而成的矩阵,各个符号的具体定义为
\begin{equation}
	g_i = \begin{bmatrix}
		g_{i1}\\
		\vdots \\
		g_{ip}
	\end{bmatrix},
	F = \begin{bmatrix}
		F & &\\
		&\ddots &\\
		& & F
	\end{bmatrix},
	M_i = F\begin{bmatrix}
		S_{i1}\\
		\vdots \\
		S_{ip}
	\end{bmatrix},
	Q=\begin{bmatrix}
		P & &\\
		&\ddots &\\
		& & P
	\end{bmatrix},
	\eta_i = \begin{bmatrix}
		\eta_{i1}\\
		\vdots \\
		\eta_{ip}
	\end{bmatrix}
\end{equation}
如果我们同时用两张切片图像进行重建,此时的重建模型为
\begin{equation}\label{multiSlice}
	\{\hat{u_1}, \hat{u_2}\} = \underset{u_1, u_2}{argmin}\left\{ \frac{1}{2} \left \Vert 
	\begin{bmatrix}
		Q_1 & 0\\
		0 & Q_2
	\end{bmatrix}
	\begin{bmatrix}
		M_1 & 0 \\
		0 & M_2 \\
	\end{bmatrix}
	\begin{bmatrix}
		u_1 \\
		u_2
	\end{bmatrix}
	- \begin{bmatrix}
		g_1 \\
		g_2
	\end{bmatrix}
	 \right\Vert_2^2 + 
	 \left\Vert
	 \Gamma W_{3D} 
	 \begin{bmatrix}
	 	u_1 \\
	 	u_2
	 \end{bmatrix}
	 \right\Vert_1
	 \right\}
\end{equation}
令 $u = [u_1 \quad u_2]^T, g = [g_1 \quad g_2]^T$,此时(\ref{multiSlice}) 可以写成
\begin{equation} \label{msmodel}
	\hat{u} = \underset{u}{argmin} \{\frac{1}{2} \Vert QMu-g \Vert_2^2 + \Vert \Gamma W_{3D}u \Vert_1\}
\end{equation}
其中$\Gamma$ 为正则化参数矩阵,对角线上元素非0, $W_{3D}$是3维紧框架变换矩阵。
\section{Algorithm}
\par 令$K = QM, A = W_{3D}$, 并且定义函数$f$和$h$分别为:
\begin{equation}
	f(u) = \frac{1}{2}\Vert Ku-g \Vert_2^2, h(u) = \Vert \Gamma u\Vert_1
\end{equation}
因此(\ref{msmodel})可以写成
\begin{equation}\label{msmodel2}
	\underset{u}{min} \{f(u) + h(Au)\}
\end{equation}
此时用于求解模型 (\ref{msmodel2})的PD3O算法步骤为

\begin{align}
	s^{k+1}&=prox_{\delta h^*}((I-\gamma \delta AA^T)s^k + \delta A(u^k - \gamma \nabla f(u^k))) \\
	u^{k+1}& = u^k - \gamma \nabla f(u^k) -\gamma A^T s^{k+1} 
\end{align}
其中 $\nabla f(u) = K^T(Ku - g)$, 因此$f$的梯度为 $\Vert K \Vert_2^2-Lipschitz $连续函数。
\newpage
\bibliographystyle{unsrt}
\bibliography{reference}

\end{document}
