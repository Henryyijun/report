%\documentclass[journal]{IEEEtran}
%\documentclass[12pt,onecolumn]{IEEEtran}
%\documentclass[11pt]{article}
%\usepackage[margin=1in]{geometry}
\documentclass[preprint]{elsarticle}
%\usepackage{booktabs,dcolumn}
\usepackage{geometry,graphicx,marvosym}
\usepackage{amsmath,amsthm}
\usepackage{amsfonts}
\usepackage[usenames,dvipsnames]{pstricks}
\usepackage{pst-plot,pstricks-add}
%\usepackage{graphicx,times,amsmath,amssymb,multirow,subfigure}
\usepackage{graphicx,times,amsmath,amssymb,multirow}
\usepackage{url}
\usepackage{stfloats}
\usepackage{amsfonts,rotating}
\usepackage{color}
\usepackage{verbatim,multirow}


\newcommand{\vertiii}[1]{{\left\vert\kern-0.25ex\left\vert\kern-0.25ex\left\vert #1
		\right\vert\kern-0.25ex\right\vert\kern-0.25ex\right\vert}}

%\usepackage{tikz}
%\usetikzlibrary{graphs,graphs.standard}
%\usetikzlibrary{shapes,chains,scopes,arrows,decorations.pathmorphing,backgrounds,positioning,fit,petri}


% setting dimension of the paper
\textwidth 7.0true in
\textheight 8.9 true in
\topmargin=-20pt
\headheight=6pt
\headsep=2pt
\oddsidemargin -0.3true in
\evensidemargin -0.4true in




\newtheorem{theorem}{Theorem}
\newtheorem{lemma}{Lemma}
\newtheorem{algorithm}{Algorithm}
\newtheorem{definition}{Definition}
\newtheorem{proposition}{Proposition}

\newcommand{\dtcwt}{\operatorname{DT-\mathbb{C}WT}}
\newcommand{\tpctf}{\operatorname{TP-\mathbb{C}TF}}
\newcommand{\ctf}{\operatorname{\mathbb{C}TF}}


\newcommand{\tr}[1]{\textcolor{red}{#1}}
\newcommand{\tb}[1]{\textcolor{blue}{#1}}

\newcommand{\mO}{{\mathcal{T}}}
\newcommand{\C}{\mathbb{C}}    %complex number field
\newcommand{\N}{\mathbb{N}}    %natural numbers
\newcommand{\R}{\mathbb{R}}    %real number field
\newcommand{\Z}{\mathbb{Z}}    %integers
\newcommand{\imag}{\mathrm{i}} % imaginary unit

\newcommand{\dR}{\mathbb{R}^d}
\newcommand{\dT}{\mathbb{T}^d}
\newcommand{\dZ}{\mathbb{Z}^d}
\newcommand{\dlp}[1]{l_{#1}(\mathbb{Z}^d)}
\newcommand{\td}{\boldsymbol{\delta}}  %Dirac/Kronicker sequence
\newcommand{\bp}{\begin{proof}}
	\newcommand{\ep}{\hfill  \end{proof} }
\newcommand{\be}{ \begin{equation} }
	\newcommand{\ee}{ \end{equation} }
\newcommand{\dLp}[1]{L_{#1}(\mathbb{R}^d)}
\newcommand{\prm}{P}           %projection matrix
\newcommand{\wh}{\widehat}
\renewcommand{\le}{\leqslant}
\renewcommand{\ge}{\geqslant}
\newcommand{\bs}{\backslash}
\newcommand{\ol}{\overline}
\newcommand{\vk}{\mathsf{k}}
\newcommand{\la}{\langle}
\newcommand{\ra}{\rangle}
\newcommand{\tp}{\mathsf{T}}  %transpose
\newcommand{\conj}{\overline}
\newcommand{\supp}{\mathrm{supp}}
\newcommand{\setsp}{\;:\;}     %set separator
\newcommand{\sd}{\mathcal{S}}  %subdivision operator S
\newcommand{\tz}{\mathcal{T}}  %transition operator T

\newcommand{\wt}{\widetilde}
\renewcommand{\le}{\leqslant}
\renewcommand{\ge}{\geqslant}
\newcommand{\er}{\eqref}

\newcommand{\gep}{\varepsilon}
\newcommand{\eps}{\epsilon}
\newcommand{\gl}{\lambda}
\newcommand{\gL}{\Lambda}
\newcommand{\gd}{\delta}
\newcommand{\DAS}{\mathrm{DAS}}
\newcommand{\UDAS}{\mathrm{UDAS}}
\newcommand{\DHF}{\mathrm{DHF}}
%\newtheorem{lemma}{Lemma}
%\newtheorem{theorem}[lemma]{Theorem}
\newtheorem{example}{Example}

%\renewcommand{\theequation}{\thesection.\arabic{equation}}
%\renewcommand{\thefigure}{\thesection.\arabic{figure}}
%\renewcommand{\thetable}{\thesection.\arabic{table}}

%\numberwithin{equation}{section}
%\numberwithin{lemma}{section}
%\numberwithin{figure}{section}
%\numberwithin{example}{section}

%\bibliographystyle{elsarticle-num}




\newcommand{\xz}[1]{\textcolor{magenta}{\bf #1}}
\begin{document}
	\section{Reconstruction model} 
	The $\ell$-th  coil $k$-space data $g_\ell$  is modeled as follows:
	\begin{equation*}\label{eq:SENSE-ObsModel}
		g_\ell = \mathcal{P}\mathcal{F}\mathcal{S}_\ell u + \eta_\ell,\: \ell =1,\cdots, p,
	\end{equation*}
	where $\eta_\ell$ is the additive noise, $\mathcal{S}_\ell$ the diagonal sensitivity matrix, $\mathcal{F}$ is the discrete Fourier transform matrix  and
	$\mathcal{P}$, called sampling matrix,  is a diagonal matrix with $0$ and $1$, $u$ is target slice image.
	\tr{In real application, the sensitivity $\mathcal{S}_\ell$  and target image $u$ are unknown variables. We adopt an alternative scheme to solve problem, and need to fix one and then solve another.}
	
	\subsection{Update image $u$} When the sensitivity $\mathcal{\hat{S}}_{\ell}$ is given, we want to solve the target image $u$.
	We combine all the coil data $g_{\ell}$ and have a compact formula:
	\begin{center}\label{eq:SENSE-ObsModel_Toget}
		$g=Q_{p}Mu+\eta$,
		\par\end{center}
	where $g$ is the stacked $k$-space data, $M$ is the composition of
	$\mathcal{F}$ and $\mathcal{S}_{\ell}$, and $Q_{p}$ is the concatenation of $ \mathcal{P}$:
	%\begin{center}
	$$ g:=\left[\begin{array}{c}
		g_{1}\\
		\vdots\\
		g_{p}
	\end{array}\right],\mathcal{F}_p:=\left[
	\begin{array}{ccc}
		\mathcal{F} & \: & \: \\
		\: & \ddots & \: \\
		\: & \:  & \mathcal{F} \\
	\end{array}
	\right], M=\mathcal{F}_p\left[\begin{array}{c}
		\mathcal{\hat{S}}_{1}\\
		\vdots\\
		\mathcal{\hat{S}}_{p}
	\end{array}\right],Q_{p}=\left[\begin{array}{ccc}
		\mathcal{P} &  & \\
		& \ddots & \\
		& \ &  \mathcal{P}
	\end{array}\right],\eta=\left[\begin{array}{c}
		\eta_{1}\\
		\vdots\\
		\eta_{p}
	\end{array}\right]$$
	The proposed optimization model by  tight framelet regularization  for SENSE-based MRI reconstruction is:
	\begin{equation}\label{mod:SENSE-ObsModel_3DReg}
		\hat{u}=\arg\min_{u}\left\{ \frac{1}{2}\| Q_{p}Mu-g \|_{2}^{2}+\| \Gamma W \mathcal{F}_p^{-1}(Q_{o}Mu+g)\| _{1}\right\}
	\end{equation}
	Where $Q_{o}= I-Q_{p}$ is the sampling matrix at the missing positions,  $\Gamma$ is a diagonal matrix with non-negative diagonal elements and the $W$ is a transformation matrix by 3-dimension and 2-dimension tight framelet system. 
	\par Let $x$ be a multi coil image of size $m \times n \times c$ in $\mathcal{C}^{m \times n \times c}$, therefore the  tight frame transform using the filters for $u \in \mathcal{C}^{m \times n \times c}$ is given by
	
	
	\begin{equation*}
		Wx = \begin{bmatrix}
			W_{3D, l}\circledast x & W_{3D, x}\circledast x & W_{3D, y}\circledast x & W_{2D}W_{3D, z}\circledast x & W_{3D, xy}\circledast x
		\end{bmatrix}^T
	\end{equation*}
	where the $W_{3D, l}$ denotes the lowpass filter of the 3-D transformation, $W_{2D}$ is the 2-D transformer matrix, while others represent the highpass filter at different directions of the 3-D transformation.
	
	\section{Algorithm}
	Let $K\doteq Q_{p}M$, $A\doteq W\mathcal{F}_p^{-1}Q_{o}M$ and $b \doteq W\mathcal{F}_p^{-1} g$,
	and identify the functions $f$, and $p$ as follows:
	\begin{equation}\label{identify3function:1}
		f(u)=\frac{1}{2}\|Ku-g\|^2, \quad p(s)=\| \Gamma (s+b)\| _{1}
	\end{equation}
	Our optimization model \eqref{mod:SENSE-ObsModel_3DReg} is written into
	the 
	\begin{equation}\label{identify3function:2}
		\min_{u}\left\{ \frac{1}{2}\| Ku-g \|_{2}^{2}+ p(Au)\right\}. 
	\end{equation}
	 
	The  PD3O algorithm is provided as:
	\begin{subequations}
		\begin{align*}
			s^{k+1}&=\mathrm{prox}_{\delta p^*}\left((I-\gamma\delta AA^\top)s^k+\delta A(u^k-\gamma \nabla f(u^k))\right)\\
			       &=\mathrm{prox}_{\delta p^*}\left( s^k - A  (\delta \gamma A^Ts^k - \delta(u^k-\gamma \nabla f(u^k)) \right) \\
			u^{k+1}&=u^k-\gamma \nabla f(u^k)-\gamma A^\top s^{k+1}
		\end{align*}
	\end{subequations}
	In the model \eqref{identify3function:2}, the operator $A\doteq W\mathcal{F}_p^{-1}Q_{o}M$, and its odjoint can be formulated as $A^T\doteq M^T Q_{o}^{T} \mathcal{F}_{p} W^T$, where $W^T$ denotes the adjoint matirx of the $W$. Assuming the $\omega$ obtained by the $Wx$, where $\omega$ is the tight frame coefficient of the multi coil image $x$ and satisfies the perfect reconstruction formula, i.e.,
	
	\begin{equation*}
		 x = W^T W x = W^T \omega
	\end{equation*}
	\begin{eqnarray*}
		W^T \omega = W_{3D, l}^{T}\circledast \omega_l + W_{3D, x}^{T}\circledast \omega_x + W_{3D, y}^{T}\circledast \omega_y + W_{3D, z}^{T} W_{2D}^{T}\circledast \omega_z+ W_{3D, xy}^{T} \circledast \omega_{xy}
	\end{eqnarray*}
	
\end{document}