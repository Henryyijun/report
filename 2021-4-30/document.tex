\documentclass[11pt]{article}
\usepackage[UTF8]{ctex}
\usepackage{geometry,graphicx,marvosym}
\usepackage{amsmath}
\usepackage{amsfonts}
\usepackage[usenames,dvipsnames]{pstricks}
\usepackage{pst-plot,pstricks-add}
\usepackage{graphicx,times,amsmath,amssymb,multirow}
\usepackage{url}
\usepackage{stfloats}
\usepackage{amsfonts,rotating}
\usepackage{color}
\usepackage{verbatim,multirow}
\bibliographystyle{IEEEtran}

\textwidth 7.0true in
\textheight 8.9 true in
\topmargin=-20pt
\headheight=6pt
\headsep=2pt
\oddsidemargin -0.3true in
\evensidemargin -0.4true in

\title{pMRI研究现状}
\author{Henry}
\begin{document}
	\maketitle 
	
	\section{背景}
	\par 磁共振成像(Magnetic Resonance Imaging,MRI)是一项无创的医学成像技术,用于检测人体组织中是否存在病变等情况,因此在临床上被广泛引用。但是由于其成像速度过于缓慢,并行磁共振成像(Parallel Magnetic Resonance Imaging,pMRI)用以加速成像,它同时使用多个线圈采集磁共振信号,使用下采样的方法只采集部分数据,最后使用重建算法从欠采样的数据中得到目标图像。
	\section{国内外研究现状}
	\par 使用重建算法从欠采样的数据中得到目标图像,主要可以分为两大类方法:(1)基于图像域的算法,例如Sensitivity encoding(SENSE)\cite{pruessmann1999sense};(2)基于频率域的算法,例如Generalized Autocalibrating Partially Parallel Acquisitions(GRAPPA) \cite{griswold2002generalized}。
		
	\subsection{基于图像域的重建方法}
	\par
	
	
	\subsection{基于频率域的重建方法} 
	
	\newpage
	\bibliography{reference}
\end{document}