\documentclass[11pt]{article}
\usepackage[UTF8]{ctex}
\usepackage{geometry,graphicx,marvosym}
\usepackage{amsmath}
\usepackage{amsfonts}
\usepackage[usenames,dvipsnames]{pstricks}
\usepackage{pst-plot,pstricks-add}
\usepackage{graphicx,times,amsmath,amssymb,multirow}
\usepackage{url}
\usepackage{stfloats}
\usepackage{amsfonts,rotating}
\usepackage{color}
\usepackage{verbatim,multirow}
\bibliographystyle{IEEEtran}

\textwidth 7.0true in
\textheight 8.9 true in
\topmargin=-20pt
\headheight=6pt
\headsep=2pt
\oddsidemargin -0.3true in
\evensidemargin -0.4true in
\setlength{\parskip}{0.3em}

\title{pMRI研究现状}
\author{Henry}
\begin{document}
	\maketitle 
	
	\section{背景}
	\par 磁共振成像(Magnetic Resonance Imaging,MRI)是一项无创的医学成像技术,用于检测人体组织中是否存在病变等情况,因此在临床上被广泛引用。但是由于其成像速度过于缓慢,并行磁共振成像(Parallel Magnetic Resonance Imaging,pMRI)用以加速成像,它同时使用多个线圈采集磁共振信号,使用欠采样的方法只采集部分数据,最后使用重建算法从欠采样的数据中重建得到目标图像。
	\section{国内外研究现状}
	\par 使用重建算法从欠采样的数据中得到目标图像,主要可以分为两大类方法:(1)基于图像域的算法,例如Sensitivity encoding(SENSE)\cite{pruessmann1999sense};(2)基于$k$-空间的算法,例如Generalized Autocalibrating Partially Parallel Acquisitions(GRAPPA) \cite{griswold2002generalized}。
		
	\subsection{基于图像域的重建方法}
	\par pMRI系统使用一组线圈阵列进行同时采集数据,线圈阵列由多个线圈组成,用于采集不同空间位置上的信息,使得越接近线圈的区域上信号会更强,这种不同空间位置导致的信号强弱被称为线圈的敏感度。通常情况下,线圈的敏感度未知,而基于图像域的重建方法首先需要明确知道敏感度矩阵,因此重建图像的质量十分依赖于敏感度信息。
	\par 在图像域中广泛使用的是SENSE类重建方法,其将pMRI图像重建问题视作一个逆问题进行考虑,在最小二乘意义下最小化测量值与真实值之间的误差。在一般情况下,pMRI图像重建问题是一个病态问题,因此需要进行正则化处理,广泛使用的正则化方法有 $l_1$正则化,$l_2$正则化等。
	\par 近年来,稀疏约束和低秩约束等正则化方法被广泛应用到pMRI图像重建问题中,Lustig等人将压缩感知应用于MRI图像重建中(CS MRI)\cite{lustig2007sparse},其实验表明该方法可以较好的去除图像伪影,得到较高质量的重建图像。Martin Uecker等人提出了一种估计敏感度矩阵的方法\cite{uecker2014espirit},该方法使用自校准数据构造得到自校准矩阵,对校准矩阵进行奇异值分解,之后使用右奇异矩阵进行构造得到filters,对filter傅里叶逆变换后进行特征值分解,只取特征值为1的对应的特征向量构造敏感度矩阵。紧接着他们提出了“soft” SENSE模型,使用多组敏感度矩阵和多组图像分量用于重建,最后使用具有稀疏性的$l_1$范数得到了$l_1$-ESPIRiT方法。Liu等人提出了projectediterative soft-thresholding algorithm(pISTA),与其加速版本pFISTA\cite{7448403},所提出的算法利用了MRI图像在紧框架表示下的冗余性,且该算法只有一个需要调节的参数。pFISTA虽然算法简洁且参数简单,但是其收敛性只适用于单线圈,而不适用于并行磁共振成像。Zhang等人对Liu等人提出的pFISTA进行了改进,分析了该算法在多线圈并行磁共振成像下的收敛性,提出了pFISTA并行成像版本\cite{ZHANG2021101987}。同时将该算法应用于两个经典重建模型中,SENSE\cite{pruessmann1999sense}和SPIRiT\cite{lustig2010spirit},提供了pFISTA求解这两个模型的最优参数。
	\par dads\cite{8017620} asdasd \cite{8428648}
	
	
	
	\subsection{基于$k$-空间的重建方法} 
	
	\newpage
	\bibliography{reference}
\end{document}