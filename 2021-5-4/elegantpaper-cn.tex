%!TEX program = xelatex
% 完整编译: xelatex -> bibtex -> xelatex -> xelatex
\documentclass[lang=cn,11pt,a4paper,cite=numbers]{elegantpaper}

\title{pMRI研究现状}
\author{Henry}
%\institute{\href{https://elegantlatex.org/}{Elegant\LaTeX{} 项目组}}
\date{\zhtoday}
% 本文档命令
\usepackage{array}

\newcommand{\ccr}[1]{\makecell{{\color{#1}\rule{1cm}{1cm}}}}
\begin{document}

\maketitle
\section{背景}
\par 磁共振成像(Magnetic Resonance Imaging,MRI)是一项无创的医学成像技术,用于检测人体组织中是否存在病变等情况,因此在临床上被广泛引用。但是由于其成像速度过于缓慢,并行磁共振成像(Parallel Magnetic Resonance Imaging,pMRI)用以加速成像,它同时使用多个线圈采集磁共振信号,使用欠采样的方法只采集部分数据,最后使用重建算法从欠采样的数据中重建得到目标图像。
\section{国内外研究现状}
\par 使用重建算法从欠采样的数据中得到目标图像,主要可以分为两大类方法:(1)基于图像域的算法,例如Sensitivity encoding(SENSE)\cite{pruessmann1999sense};(2)基于$k$-空间的算法,例如Generalized Autocalibrating Partially Parallel Acquisitions(GRAPPA) \cite{griswold2002generalized}。

\subsection{基于图像域的重建方法}
\par pMRI系统使用一组线圈阵列进行同时采集数据,线圈阵列由多个线圈组成,用于采集不同空间位置上的信息,使得越接近线圈的区域上信号会更强,这种不同空间位置导致的信号强弱被称为线圈的敏感度。通常情况下,线圈的敏感度未知,而基于图像域的重建方法首先需要明确知道敏感度矩阵,因此重建图像的质量十分依赖于敏感度信息。

\par 在图像域中广泛使用的是SENSE类重建方法,其将pMRI图像重建问题视作一个逆问题进行考虑,在最小二乘意义下最小化测量值与真实值之间的误差。在一般情况下,pMRI图像重建问题是一个病态问题,因此需要进行正则化处理,广泛使用的正则化方法有 $l_1$正则化,$l_2$正则化等。但是在现有的SENSE类方法中,从较少的ACS数据中估计得到的敏感度矩阵仍然不太准确,例如Ying等人提出的JSENSE\cite{ying2007joint},该算法将MRI图像重建问题转为线圈敏感度和图像的联合估计问题,使用优化迭代算法进行求解。相比于JSENSE,ESPIRiT\cite{uecker2014espirit}将GRAPPA和SENSE的思想融合在一起,使用滑动窗口构建校准矩阵,并且对校准矩阵进行奇异值分解和特征值分解的方法,从ACS中估计出较为准确的敏感度矩阵,进而提出soft SENSE方法,使用多张敏感度矩阵的方式进行重建,再对多张图像分量使用$l_1$范数约束,最后利用共轭梯度法求解,将得到的多张图像分量使用SOS进行合并。

\par 随着压缩感知(Compressed Sensing, CS)等理论的研究深入,Lustig等人发现一些MRI图像中的像素表示是较为稀疏的,如果使用适当的重建算法,具有稀疏性的MRI图像将可以从任意采样的k-空间数据中得到恢复,因此提出了Sparse-MRI\cite{lustig2007sparse},该方法也奠定了CS与MRI结合的算法框架。Liang等人将CS-MRI和SENSE方法结合一起,提出了CS-SENSE\cite{liang2009accelerating}。

\par Liu等人提出了projectediterative soft-thresholding algorithm(pISTA),与其加速版本pFISTA\cite{7448403},所提出的算法利用了MRI图像在紧框架表示下的冗余性,且该算法只有一个需要调节的参数。pFISTA虽然算法简洁且参数简单,但是其收敛性只适用于单线圈,而不适用于并行磁共振成像。Zhang等人对Liu等人提出的pFISTA进行了改进,分析了该算法在多线圈并行磁共振成像下的收敛性,提出了pFISTA并行成像版本\cite{ZHANG2021101987}。同时将该算法应用于两个经典重建模型中,SENSE\cite{pruessmann1999sense}和SPIRiT\cite{lustig2010spirit},提供了pFISTA求解这两个模型的最优参数。

\subsection{基于$k$-空间的重建方法} 
\par 基于k-空间的重建方法与基于图像域方法的主要区别为,不需要明确知道线圈敏感度的信息,其信息隐含在k-空间的数据中。k-空间重建方法将pMRI重建问题视为一个矩阵



\newpage
\bibliographystyle{unsrt}
\bibliography{reference}





\end{document}
